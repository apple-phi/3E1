% If you need extra functionality,
% consider using the `scrartcl` class from KOMA-Script
\documentclass[10pt]{article}

\usepackage[utf8]{inputenc}
\usepackage{latexsym,amsfonts,amssymb,amsthm,amsmath}
\usepackage{graphicx}
% \graphicspath{ {../data} }

% Force figures to stay in their section or subsection
% https://tex.stackexchange.com/a/32623
\usepackage[section]{placeins}
% https://tex.stackexchange.com/a/118667
\makeatletter
\AtBeginDocument{%
  \expandafter\renewcommand\expandafter\subsection\expandafter{%
    \expandafter\@fb@secFB\subsection
  }%
}
\makeatother

\usepackage{pdfpages} % For including PDFs like the cover-sheet


% Link references to figures
% https://stackoverflow.com/a/21251925
\usepackage{hyperref}
\usepackage{caption}

\title{3E1: On how proposed \textit{Schuldenbremse} relaxation could impact the financial performance of Volkswagen AG}
\author{Lucas Ng\footnote{ln373@cam.ac.uk}}
\date{9th December 2024}

\usepackage{csquotes}
\usepackage[notes, backend=bibtex]{biblatex-chicago}
\addbibresource{refs.bib}


\begin{document}

\maketitle

% Topic 1: Reflect on how a recent or proposed UK/EU government economic policy
% change will impact on the financial performance of a named business of your choice.
% In answering the question, you may want to consider (non-exhaustive list):
% 1. A description of the nature of the business you are analysing.
% 2. A discussion of how the firm makes profits/a surplus now and how that may change
% in the near future.
% 3. An economic analysis of the key strategic choices that a firm of this type has to make
% within its market(s).
% 4. A description of the policy change you are analysing.
% 5. An economic analysis, using relevant concepts and diagrams, of the potential effect
% of the policy change.
% 6. A discussion of how the business could best respond to the recent or proposed policy
% change and what strategy it might adopt to enhance positive or mitigate negative
% effects.
% 7. References to relevant parts of the course and other references from the economic
% literature.

\section{Introduction}
Volkswagen AG (VW) is a German publicly-traded multinational conglomerate manufacturer of vehicles, engines and turbomachinery whose dominance in the automotive industry is underpinned by its diverse portfolio of brands including Volkswagen, Audi, Porsche, and Lamborghini. Its recent challenges include the transition to electric vehicles, the still-ongoing diesel emissions scandal, and the global semiconductor shortage.

Amid Germany's slow but steady descent into economic malaise and political paralysis, the recent and well-documented collapse of the incumbent Chancellor Scholz's three-way coalition government has prompted fresh questions about the necessity of the \textit{Schuldenbremse}, a constitutional debt brake that limits the federal government's structural deficit to 0.35\% of GDP in addition to the EU-wide fiscal restraints on overall budget deficits.\@
The \textit{Schuldenbremse} has been a cornerstone of German fiscal policy since its introduction in 2009 by ex-Chancellor Merkel,
and while its proponents maintain that it has been essential in cementing Germany's historical position as fiscal policeman, its critics argue that it has stifled public investment and economic growth.

This debt brake was instrumental in last year's  cancellation of proposed investments in the green energy transition which would have been significantly beneficial in VW's ability to fend off market competition from China. Considering the threat of a potential increase in Chinese competition in electric vehicle markets diverted to the Eurozone due to Trumpian tariffs, the relaxation of the \textit{Schuldenbremse} could be a necessity to Volkswagen's continued financial performance.

\newpage

\section{Volkswagen AG as a firm}

% • Multinational conglomerate manufacturer headquartered in Wolfsburg, Germany
% • Founded in 1937, initially to produce the Beetle
% • Largest automaker by sales in 2016-2019, selling 10.9 million vehicles
% • Largest company in the European Union and largest car manufacturer globally by revenue in 202
% • Diverse brand portfolio:
% Passenger cars: Volkswagen, Audi, Porsche, and Lamborghini
% Motorcycles: Ducati
% Commercial vehicles: Volkswagen Commercial Vehicles
% Heavy commercial vehicles: Traton (International Motors, MAN, Scania, Volkswagen Truck & Bus)
% • Operations in approximately 150 countries with 100 production facilities across 27 countries
% • Two primary divisions: Automotive and Financial Services
% • Significant presence in China through joint ventures: FAW-Volkswagen, SAIC Volkswagen, and Volkswagen Anhui

Volkswagen AG is a multinational conglomerate manufacturer headquartered in Wolfsburg, Germany. Founded in 1937, it initially produced the Beetle and has since grown to become the largest automaker by sales in 2016-2019, selling 10.9 million vehicles. It is the largest company in the European Union and the largest car manufacturer globally by revenue in 2023. Volkswagen has a diverse brand portfolio, including passenger cars such as Volkswagen, Porsche, and Lamborghini, motorcycles, and its business in commercial vehicles under Traton SE\autocite{vwgroupHistory1999}.

Volkswagen operates in approximately 150 countries with 100 production facilities across 27 countries, with a significant presence in China through joint ventures such as SAIC Volkswagen which have been instrumental in bolstering its revenue. VW is China's best-selling foreign automaker---correspondingly, the Chinese markets generate at least half of VW Group's profits\autocite{whiteVWAuditXinjiang2024}.

\

% \section{Profit generation}
% Current profit sources:
% • Diverse product range across multiple market segments and price points
% • Global presence, particularly strong in Europe and China
% • Revenue from vehicle sales, after-sales services, and financial services
% • Economies of scale in production and research & development

% Potential changes in profit generation:
% • Transition to electric vehicles (EVs) affecting profit margins
% • Increased competition in the EV market, especially from Chinese manufacturers
% • Potential impact of semiconductor shortages on production and sales
% • Ongoing costs related to the diesel emissions scandal
% • Shift towards direct sales of semiconductor chips to secure supply chain

\noindent Volkswagen generates profits through a diverse product range across multiple market segments and price points. Its global presence, particularly strong in Europe and China, has been instrumental in its revenue generation. Volkswagen's revenue comes from vehicle sales, after-sales services, and financial services. The company benefits from economies of scale in production and R\&D, allowing it to maintain a competitive edge\autocite{VolkswagensEarningsDisappoint09:07:19+01:00}.

However, the transition to electric vehicles (EVs) is expected to affect profit margins, as the costs associated with EV production are higher than those of traditional combustion engine vehicles. Increased competition in the EV market, especially from Chinese manufacturers, poses a threat to Volkswagen's market share and profitability. The ongoing semiconductor shortage has also impacted production and sales, leading to reduced revenue and profits. Additionally, the costs related to the diesel emissions scandal continue to affect Volkswagen's financial performance. To mitigate the impact of the semiconductor shortage, Volkswagen has shifted towards the direct purchase of semiconductor chips to secure its supply chain and maintain production levels\autocite{VWGroupBuy2023}.

\

The auto industry is oligopolistic, with a few large firms dominating the market, of which VW is one. The industry is capital-intensive, requiring significant R\&D investment, production facilities, and marketing. VW's global presence and diverse product portfolio have been instrumental in its success\autocite{vwgroupHistory1999}. However, an increasing trend towards a multipolar flavour of globalisation has seen the rise of Chinese automakers, particularly in the EV market, posing a significant challenge to the well-established VW\autocite{VolkswagensEarningsDisappoint09:07:19+01:00}.

\

% \section{Strategic choices}
% • Balancing investment between traditional combustion engines and electric vehicles
% • Managing the transition to EVs while maintaining profitability
% • Addressing competition in key markets, particularly China
% • Securing supply chains, especially for critical components like semiconductor chips
% • Expanding into emerging markets (e.g., India, Africa) to offset potential losses in established markets
% • Investing in research and development for new technologies and innovations
% • Considering joint ventures or acquisitions to strengthen market position
% • Balancing cost reduction efforts with maintaining product quality and brand reputation

\noindent Volkswagen faces several strategic choices in its market. It must balance investment between traditional combustion engines and electric vehicles to meet consumer demand while maintaining profitability. Managing the transition to EVs is crucial for VW's long-term success, as the automotive industry shifts towards sustainable transportation. Addressing competition in key markets, particularly China, is essential for VW to maintain its market share and revenue. Securing supply chains, especially for critical components like semiconductor chips, is vital to ensure uninterrupted production and sales. Expanding into emerging markets, such as India and Africa, can help offset potential losses in established markets and diversify Volkswagen's revenue streams. Investing in research and development for new technologies and innovations is necessary to stay competitive in the automotive industry.

Considering joint ventures or acquisitions to strengthen its market position and expand its product portfolio is another strategic choice for Volkswagen. By balancing cost reduction efforts with maintaining product quality and brand reputation, VW may be able to remain competitive and profitable in the automotive market\autocite{VolkswagensEarningsDisappoint09:07:19+01:00}.

\section{Reform of the \textit{Schuldenbremse}}
% • Potential relaxation of the Schuldenbremse (German debt brake)
% • Current policy: Limits federal government's structural deficit to 0.35% of GDP
% • Proposed change: Easing or removing this limit to allow for increased public spending
% • Context: Economic challenges in Germany, political instability, and calls for increased investment in green energy transition
% • Implications: Potential for increased government investment in infrastructure, green technologies, and economic stimulus

The \textit{Schuldenbremse}, or German debt brake, is a constitutional fiscal restraint that limits the German federal government's structural deficit to 0.35\% of GDP.\@ The debt brake was introduced in 2009 by ex-Chancellor Merkel as part of Germany's response to the global financial crisis and has been a cornerstone of German fiscal policy since then. It is designed to ensure fiscal discipline and prevent excessive government borrowing. Proponents view it as important in cementing Germany's position as Eurozone fiscal policeman. However, critics argue that the debt brake has stifled public investment and economic growth, particularly in the context of the economic challenges facing Germany and the wider Eurozone\autocite{pitelWillFriedrichMerz2024}.

In light of the recent collapse of the incumbent Chancellor Scholz's three-way coalition government over budgetary disputes\autocite{chazanGermanCoalitionGovernment2024}, there have been calls for the relaxation of the \textit{Schuldenbremse} to allow for increased public spending. The proposed change would ease or remove the limit on the federal government's structural deficit, enabling increased investment in infrastructure, green technologies, and economic stimulus\autocite{pitelWillFriedrichMerz2024}.

Following the implosion of the coalition, the bund market quickly priced this policy change in, with a sell-off in 10 year bunds and a corresponding increase in yields and decrease in bund price. In particular, Bund/euro swap spreads, which, due to the relative scarcity of bunds, have long traded at positive values (unlike most other developed markets), turned negative for the first time.

However, this price change reverted somewhat due to uncertainty in whether the policy change will make it through the \textit{Bundestag}\autocite{chazanGermanBondInvestors2024}. With snap elections called for February 2025, the future of the \textit{Schuldenbremse} remains uncertain.

\

% \section{Economic analysis}
% • Potential increase in government spending:
% The 2024 federal budget allocates around €53 billion for investment spending
% Relaxing the debt brake could significantly increase this figure
% • Impact on EV transition:
% Germany's switch to EVs could cost the auto industry 186,000 jobs by 2035
% 75,000 jobs have already been lost since 2019, with 46,000 net job losses
% Increased government spending could help mitigate these job losses through retraining programs and industry support
% • Infrastructure investment:
% The Climate and Transformation Fund earmarks €49.1 billion for climate-neutral transformation in 2024
% Relaxing the debt brake could boost this figure, potentially accelerating EV charging infrastructure development
% • Research and development support:
% The 2024 budget allocates €21.5 billion to Education and Research
% Additional funding could help VW compete with Chinese EV manufacturers
% • Competitiveness measures:
% The Growth Opportunities Act (Wachstumschancengesetz) creates tax incentives for investment and innovation
% Relaxing the debt brake could allow for more aggressive incentives
% • Energy cost reduction:
% Current policy reduces electricity duty for manufacturing companies to €0.5 per MWh
% Additional funding could further reduce energy costs, boosting VW's competitiveness
% • Economic growth implications:
% Germany's GDP growth is forecast near zero for 20244
% Increased government spending could stimulate economic growth, potentially boosting domestic demand for VW products
% • Industrial output concerns:
% German industrial production is down by almost 3% year-on-year as of August 20246
% VW's production could benefit from broader industrial support measures
% • Fiscal policy shift:
% Current policy aims to return to pre-crisis expenditure levels
% Relaxing the debt brake would mark a significant shift in fiscal strategy
% • Investment in digitalisation:
% The government emphasises digitalizing the economy, infrastructure, and education
% This could help VW's transition to software-defined vehicles and smart manufacturing
% • Potential drawbacks:
% ZEW study shows additional budgetary funds are mainly used for consumptive spending rather than investment (ratio of 3:1)1
% Risk of inefficient allocation of resources without proper prioritisation
% • Currency effects:
% Increased government borrowing could potentially weaken the Euro
% This might improve VW's export competitiveness outside the Eurozone
% • Long-term fiscal sustainability:
% Relaxing the debt brake could raise concerns about Germany's long-term fiscal health
% This might impact investor confidence in German companies, including VW

% DELETED paragraph:
% Any measure of \textit{Schuldenbremse} relaxation would have several significant implications. The 2024 German federal budget allocates around €53 billion for investment spending\autocite{GermanStabilityProgramme}, and relaxing the debt brake could significantly increase this figure. VW cars can be considered to be a normal good with positive income elasticity of demand, hence with increased consumer spending power, VW's sales could increase.

% DELETED paragraph:
% Infrastructure investment is another area that could benefit from increased government spending. The Climate and Transformation Fund earmarks €49.1 billion for climate-neutral transformation in 2024, and relaxing the debt brake could boost this figure, potentially accelerating EV charging infrastructure development. Research and development support is crucial for Volkswagen to compete with Chinese EV manufacturers. The 2024 budget allocates €21.5 billion to Education and Research, and additional funding could help VW stay competitive in the EV market\autocite{GermanStabilityProgramme}.

\noindent Relaxing the \textit{Schuldenbremse} and increasing government spending could potentially stimulate economic growth through fiscal multipliers, though the effectiveness depends on the type of spending. Investment in infrastructure, particularly in EV charging networks and green technologies, may yield high multipliers by enhancing productivity and fostering sustainable growth, indirectly benefiting Volkswagen. However, consumptive spending, which has a lower multiplier effect, might offer limited long-term economic benefits. Therefore, while targeted fiscal expansion could boost demand and improve the business environment for VW, the overall impact hinges on efficient allocation and the broader economic context, including inflationary pressures and public debt sustainability. The 2024 budget allocates €21.5 billion to Education and Research, and additional funding could help VW stay competitive in the EV market\autocite{GermanStabilityProgramme}. Industrial output concerns, such as the 3\% year-on-year decline in German industrial production as of August 2024, could benefit from broader industrial support measures, potentially boosting VW's production\autocite{brzeskiReboundGermanIndustrial}.


Further, competitiveness measures such as the Growth Opportunities Act (\textit{Wachstumschancengesetz}) create tax incentives for investment and innovation. Relaxing the debt brake could allow for more aggressive incentives, boosting VW's competitiveness. Energy cost reduction is another potential benefit of increased government spending. Current policy reduces electricity duty for manufacturing companies to €0.5 per MWh, and additional funding could further reduce energy costs, enhancing VW's competitiveness\autocite{GermanStabilityProgramme}.

% Increased government spending could stimulate economic growth, potentially boosting domestic demand for VW products. Germany's GDP growth is forecast near zero for 2024, and increased government spending could help stimulate economic growth. Industrial output concerns, such as the 3\% year-on-year decline in German industrial production as of August 2024, could benefit from broader industrial support measures, potentially boosting VW's production\autocite{brzeskiReboundGermanIndustrial}.

Relaxing the debt brake would mark a significant shift in fiscal strategy, as current policy aims to return to pre-COVID expenditure levels. The government emphasises digitalizing the economy, infrastructure, and education, which could help VW's transition to software-defined vehicles and smart manufacturing. However, there are potential drawbacks to increased government spending. A ZEW study shows that additional budgetary funds are mainly used for consumptive spending rather than investment, with a ratio of 3:1. This raises the risk of inefficient allocation of resources without proper prioritisation\autocite{PressReleaseEasing}.

Currency effects are a final consideration, as increased government borrowing could potentially weaken the Euro. This might improve VW's export competitiveness outside the Eurozone. However, increased public debt undermines fiscal sustainability, potentially leading to higher borrowing costs and crowding out private investment. Moreover, expanded fiscal deficits may stoke inflationary pressures, particularly in an economy already grappling with supply chain disruptions and labor market constraints. These inflationary dynamics could erode real incomes and consumer spending power, paradoxically dampening the anticipated demand boost for companies like Volkswagen. Additionally, greater fiscal leniency might provoke adverse reactions in financial markets, undermining investor confidence and stability in the Eurozone. Such complexities necessitate a cautious and balanced approach to any relaxation of the fiscal rules.

\section{Response and strategy}

% • Address labor disputes and cost-cutting measures:
% Negotiate with IG Metall union to avoid extended strikes
% Reconsider proposed wage cuts and factory closures in Germany
% Explore alternatives to layoffs, potentially focusing on voluntary early retirement or reduced hours
% • Accelerate EV transition while maintaining flexibility:
% Continue development of ID family of electric vehicles
% Incorporate more plug-in hybrid models to cater to evolving consumer preferences
% Balance investment between EVs and traditional combustion engines
% • Leverage strategic partnerships:
% Capitalise on $5.8 billion joint venture with Rivian to share technology and create affordable EVs
% Utilise Rivian's expertise in electric powertrains and battery systems
% Aim to launch first models with Rivian technology by 2027
% • Adapt production and cost structure:
% Reevaluate plans to close German plants, considering Chancellor Scholz's comments
% Optimise manufacturing processes to improve efficiency without resorting to plant closures
% Explore ways to reduce production costs while maintaining German workforce
% • Address market challenges:
% Develop strategies to compete with Chinese EV manufacturers in Europe
% Focus on improving profit margins, particularly for the Volkswagen brand (currently at 2.1%)
% Adjust production targets in line with updated forecast of 9 million vehicle deliveries for 2024
% • Enhance financial performance:
% Work towards achieving the revised operating profit target of around 18 billion euros for 2024
% Implement measures to improve net cash flow in the Automotive Division
% Manage expenses related to M&A activities, including the Rivian joint venture
% • Invest in innovation and future technologies:
% Continue research and development in software-defined vehicles
% Explore advancements in autonomous driving technologies
% Invest in battery technology to improve EV range and performance
% • Strengthen market position:
% Develop strategies to regain market share in key regions, particularly in the EV segment
% Focus on customer-centric innovations to differentiate from competitors
% Adapt marketing strategies to highlight the benefits of Volkswagen's diverse powertrain options
% • Address supply chain and production issues:
% Develop strategies to mitigate the impact of semiconductor shortages
% Strengthen relationships with key suppliers to ensure stable component supply
% Explore opportunities for vertical integration in critical areas like battery production
% • Enhance sustainability efforts:
% Accelerate the transition to carbon-neutral production
% Develop and promote circular economy initiatives in vehicle manufacturing
% Align sustainability goals with potential changes in German government policies

Volkswagen has several strategic options to consider with the possibility of \textit{Schuldenbremse} reform on the horizon. Wider than just its own immediate implications, there is an ongoing labor dispute with IG Metall, Germany's most powerful trade union, over proposed wage cuts and factory closures in Germany. It would be prudent for VW to negotiate with IG Metall to avoid extended strikes and reconsider proposed cost-cutting measures to address the labor dispute.


Volkswagen should accelerate its transition to electric vehicles while maintaining flexibility in its product offerings. The company should continue the development of its ID family of electric vehicles and incorporate more plug-in hybrid models to cater to evolving consumer preferences. Balancing investment between EVs and traditional combustion engines is essential to meet consumer demand and maintain profitability\autocite{VolkswagensEarningsDisappoint09:07:19+01:00}.

Amid Donald Trump's election as the next US President from 2025, the EU member states have agreed to impose tariffs of up to 45\% on imports of Chinese electric vehicles with anti-subsidy tariffs of up to 35.3\%, on top of an existing 10\% levy. The EU tariffs will last for up to five years and range from 7.8\% for Tesla to 35.3\% for SAIC, which, as mentioned previously, operates closely with VW.\@
The move comes as Chinese companies have embarked on an aggressive expansion in Europe, where domestic carmakers are struggling to produce electric vehicles cheaply. Volkswagen Group should develop strategies to compete with Chinese EV manufacturers in Europe and focus on improving profit margins, particularly for the Volkswagen brand, which is currently at 2.1\%\autocite{inagakiEUMemberStates2024}.

Finally, Volkswagen should leverage strategic partnerships to enhance its market position and drive innovation. The company's \$5.8 billion joint venture with Rivian provides an opportunity to share technology and create affordable EVs. By utilizing Rivian's expertise in electric powertrains and battery systems, Volkswagen can aim to launch its first models with Rivian technology by 2027. This partnership could help Volkswagen stay competitive in the EV market and differentiate its offerings from competitors\autocite{VolkswagenRivianLaunch}.

\section{Conclusion}
The debate over Germany's Schuldenbremse presents both opportunities and risks for Volkswagen AG as it navigates a rapidly evolving automotive landscape. Relaxing the fiscal constraints could provide the necessary public investment to bolster green infrastructure and support the broader industrial ecosystem critical for Volkswagen's transition to electric vehicles. However, this potential boon must be weighed against the macroeconomic risks of fiscal laxity, including inflationary pressures, debt sustainability concerns, and potential market destabilisation.

Volkswagen's future hinges not only on external fiscal policy but also on its strategic adaptability. Addressing its reputation challenges, improving operational efficiency, and leveraging joint ventures for EV innovation are equally vital. Fiscal relaxation might offer temporary relief, but long-term resilience will depend on Volkswagen's ability to align its business strategies with both market trends and evolving consumer expectations. Thus, while the Schuldenbremse reform could create a more favorable environment, the onus remains on Volkswagen to capitalise effectively and sustainably on any policy shifts.

\newpage
\printbibliography

\end{document}